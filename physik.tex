\documentclass{ajc}
\usepackage[output-decimal-marker={,},per-mode=fraction,fraction-function=\tfrac]{siunitx}
\pdfsuppresswarningpagegroup=1
\usetikzlibrary{arrows, decorations.pathmorphing}


\title{Physik}
\author{Alexander Jacob}
\date{Schuljahr 2024/2025}

\begin{document}
	\section{Schwingungen und Wellen}
	
	\subsection{Teilaufgabe Physikabitur G-Kurs '22 HT}
	\paragraph{Kontext:}Ein Wagen der Masse $m$ ist an einer dehn- und stauchbaren Feder mit der Federkonstanten $D = \SI{50}{\newton\per\meter}$ befestigt. Für die Rückstellkraft gilt ein lineares Kraftgesetz. In Abbildung \ref{fig:wag_ggwl} befindet sich der Wagen in der Gleichgewichtslage, so dass keine Rückstellkraft wirkt. Der Wagen wird nun wie in Abbildung \ref{fig:wag_al} dargestellt um \SI{10}{\cm} aus der Gleichgewichtslage ausgelenkt und zum Zeitpunkt \SI{0}{\s} losgelassen.
	\begin{figure}[ht]
		\centering
		\begin{minipage}[b]{0.4\textwidth}
			\includegraphics[width=\textwidth]{ph_001_wagen_ggwl.pdf}
			\caption{Wagen in Gleichgewichtslage.}
			\label{fig:wag_ggwl}
		\end{minipage}
		\hfill
		\begin{minipage}[b]{0.4\textwidth}
			\includegraphics[width=\textwidth]{ph_002_wagen_al.pdf}
			\caption{Wagen in Auslenkung von \SI{10}{\cm}.}
			\label{fig:wag_al}
		\end{minipage}
	\end{figure}

	\subsubsection{Rückstellkraft berechnen}
	\paragraph{Aufgabe:}Berechnen Sie den Betrag der Rückstellkraft, die in der Abbildung \ref{fig:wag_al} dargestellten Situation wirkt, und zeichnen Sie einen Kraftpfeil ein, sodass deren Richtung erkennbar ist.
	\begin{equation}
		F_\text{rück} = \SI{50}{\newton\per\meter} \cdot \SI{0.1}{\meter} = \SI{5}{\newton}
	\end{equation}
	
	\subsubsection{Werte bestimmen}
	\paragraph{Aufgabe:}Das Diagramm in Abbildung \ref{fig:wag_dia} zeigt für die Bewegung des Wagens den zeitlichen Verlauf der Elongation.
	
	Bestimmen Sie die folgenden Größen:
	\begin{itemize}
		\item die Masse $m$,
		\item den maximalen Betrag $v_\text{max}$ der Geschwindigkeit,
		\item den maximalen Betrag $a_\text{max}$ der Beschleunigung des Wagens.
	\end{itemize}
	
	\begin{figure}[ht]
		\centering
		\begin{tikzpicture}
			\begin{axis}[
				xlabel={$t$ in \SI{}{\s}},
				ylabel={$s$ in \SI{}{\cm}},
				xmin=-0, xmax=5,
				ymax=12, ymin=-12,
				xtick={0.5,1.0,1.5,2.0,2.5,3.0,3.5,4.0,4.5},
				ytick={-12,-10,-8,-6,-4,-2,0,2,4,6,8,10,12},
				xmajorgrids=true,
				ymajorgrids=true,
				grid style=dashed,
				]
				\addplot [
				domain=0:5, 
				samples=300, 
				color=red,
				]
				{10 * cos(((2 * pi)/(1.6)) * deg(x))};
			\end{axis}
		\end{tikzpicture}
		\caption{Zeitlicher Verlauf der Elongation.}
		\label{fig:wag_dia}
	\end{figure}
	
	\paragraph{Masse $m$:} Aus $\omega = \frac{2\pi}{T}$ und $\omega^2 = \frac{D}{m}$ ergibt sich für die Masse $m$:
	\begin{equation}
		m = \frac{D \cdot T^2}{4\pi^2}
	\end{equation}
	
	Mit den abgelesenen Werten aus dem Diagramm \ref{fig:wag_dia} ergibt sich somit für den Betrag der Masse $m$:
	\begin{equation}
		m = \frac{\SI{50}{\newton\per\meter} \cdot (\SI{1.6}{\s})^2}{4\pi^2} = \SI{3.24}{\kg}.
	\end{equation}
	
	\paragraph{Geschwindigkeit $v_\text{max}$:} Da die Steigung des Graphen aus \ref{fig:wag_dia} z.B. zum Zeitpunkt $t = \SI{1.2}{\s}$ am größten ist, muss dort auch dessen zeitliche Ableitung $v(t)$ ein Maximum besitzen. Durch Einsetzen der Werte in das allgemeine Geschwindigkeits-Zeit-Gesetz der Federschwingung ergibt sich für den maximalen Betrag $v_\text{max}$ der Geschwindigkeit:
	\begin{equation}
		v_\text{max} = v(\SI{1.2}{\s}) = - \frac{2\pi}{\SI{1.6}{\s}} \cdot \SI{0.1}{\meter} \cdot \sin\left(\frac{2\pi}{\SI{1.6}{\s}} \cdot \SI{1.2}{\s}\right) = \SI{0.393}{\m\per\s}.
	\end{equation}
	
	\paragraph{Beschleunigung $a_\text{max}$:} Die Steigung von $v(t)$ ist zum Zeitpunkt $t = \SI{0.8}{\s}$ am größten. Daher ergibt sich für den maximalen Betrag $a_\text{max}$ der Beschleunigung des Wagens:
	\begin{equation}
		a_\text{max} = a(\SI{0.8}{\s}) = - \left(\frac{2\pi}{\SI{1.6}{\s}}\right)^2 \cdot \SI{0.1}{\meter} \cdot \cos\left(\frac{2\pi}{\SI{1.6}{\s}} \cdot \SI{0.8}{\s}\right) = \SI{1.54}{\m\per\s\squared}.
	\end{equation}
	
	\subsubsection{Zusätzliches Gewicht}
	\paragraph{Aufgabe:}In dem Moment, in dem der Wagen die maximale Elongation erreicht, wird ein Gewichtsstück aufgebracht, so dass sich die Masse des Wagens Vervierfacht. Geben Sie jeweils begründet an, ob und gegebenenfalls um welchen Faktor sich dadurch die Schwingungsdauer und der Betrag der maximalen Beschleunigung ändert.
	
	\paragraph{Schwingungsdauer:} Die Schwingungsdauer verdoppelt sich, da für $T$ gilt:
	\begin{equation}
		T = 2\pi \cdot \sqrt{\frac{m}{D}}.
	\end{equation}
	
	Wird also $m$ vervierfacht, so ändert sich $T$ um den Faktor $\sqrt{4} = 2$ und wird verdoppelt.
	
	\paragraph{Maximale Beschleunigung:} Die maximale Beschleunigung verringert sich, da $a_\text{max}$ gemäß des Beschleunigungs-Zeit-Gesetzes 
	\begin{equation}
		a(t) = -\omega^2 s_\text{max} \cdot \cos\left(\omega t\right)
	\end{equation}
	
	von $\omega^2 s_\text{max}$ abhängt. Aus der Aufgabenstellung ergibt sich, dass $s_\text{max}$ unverändert bleibt. Da $\omega^2$ gemäß $\omega^2 = \frac{D}{m}$ umgekehrt proportional zu $m$ ist, verändert sich $a_\text{max}$ um den Faktor $\frac{1}{4}$.
	
	\newpage
	
	\subsection{Wellenmaschine}
	Auf einer Wellenmaschine befinden sich 19 gekoppelte Pendel, die im Abstand von \SI{5,00}{\centi\meter} auf der x-Achse angebracht sind. Das Pendel mit der Nummer 1 wird zum Zeitpunkt $t = \SI{0}{\second}$ zu einer harmonischen Schwingung mit der Gleichung $s(t) = s_0 \cdot \sin(\omega \cdot t)$ angeregt. Diese \enquote{Störung} breitet sich längs der positiven x-Achse ungedämpft aus.
	
	\begin{figure}[ht]
		\centering
		\includegraphics[width=\textwidth]{ph_003_momentanbild_0s.pdf}
		\caption{Momentanbild zum Zeitpunkt $t = \SI{0}{\second}$}
		\label{fig:mom_0s}
	\end{figure}
	
	\begin{figure}[ht]
		\centering
		\includegraphics[width=\textwidth]{ph_004_momentanbild_1-5s.pdf}
		\caption{Momentanbild zum Zeitpunkt $t = \SI{1,5}{\second}$}
		\label{fig:mom_1.5s}
	\end{figure}
	
	\subsubsection{Bestimmen von $\lambda$ und $v$}
	Bestimmen Sie Wellenlänge und Ausbreitungsgeschwindigkeit der Welle.
	
	\begin{equation}
		\lambda = 8 \cdot \SI{0,05}{\m} = \SI{0,4}{\m}
	\end{equation}
	
	\begin{equation}
		v = \frac{6 \cdot \SI{0,05}{\m}}{\SI{1,5}{\s}} = \SI{0,2}{\m\per\s}
	\end{equation}
	
	\subsubsection{Pendelnummer bestimmen}
	Geben Sie die Nummer des Pendels an, das zum Zeitpunkt $t = \SI{2,00}{\second}$ von der Störung gerade erreicht wird.
	
	$9$
	
	\subsubsection{Maximale Elongation}
	Zum Zeitpunkt $t = \SI{4,50}{\second}$ befinden sich die Oszillatoren in einem bestimmten Schwingungszustand. Skizzieren Sie das zugehörige Momentanbild und geben Sie für diesen Zeitpunkt die Nummern der Pendel mit maximaler Elongation an.
	
	\begin{figure}[ht]
		\centering
		\includegraphics[width=\textwidth]{ph_005_momentanbild_4-5s.pdf}
		\caption{Momentanbild zum Zeitpunkt $t = \SI{4,5}{\second}$}
		\label{fig:mom_4.5s}
	\end{figure}
	
	$1,5,9,13,17$
	
	\newpage
	
	\subsection{Mechanische Wellen}
	\paragraph{Angabe:} Schallgeschwindigkeit in Luft: \SI{340}{\m\per\s}.
	
	\subsubsection{Wellenzug}
	Wie lang ist der Wellenzug (\enquote{Störung}), der mit einem \SI{300}{\milli\s} langen Geräusch in Luft erzeugt wird?
	\begin{equation}
		s = v \cdot t = \SI{340}{\m\per\s} \cdot \SI{0,300}{\s} = \SI{102}{\m}
	\end{equation}
	
	\subsubsection{Wellenmaschine}
	Der erste Oszillator (bei $x = 0$) einer Wellenmaschine werde 40 mal pro Minute auf und ab bewegt.
	
	Diese Störung breitet sich mit \SI{30}{\cm\per\s} über den Wellenträger aus. In welchem kleinstmöglichen Abstand vom ersten Oszillator schwingt ein anderer Oszillator synchron zum ersten?
	
	$v = \SI{30}{\cm\per\s}$
	
	$f = \SI{0,666}{\per\s}$
	\begin{equation}
		\lambda = \frac{v}{f} = \frac{\SI{30}{\cm\per\s}}{\SI{0,666}{\per\s}} = \SI{0,45}{\m}
	\end{equation}
	
	\subsubsection{Interferenz}
	An der Stelle $x_A = -20$ bzw. $x_B = +19$ einer in Metern skalierten x-Achse stehen zwei Lautsprecher, die jeweils gleichphasig eine Schallwelle der Frequenz \SI{440}{\hertz} in die Luft abstrahlen.
	
	\paragraph{a)} Berechne den Gangunterschied beider Wellen im Nullpunkt und zeige, dass keine vollständige Auslöschung im Nullpunkt stattfindet.
	\begin{equation}
		\Delta s = \left|s_2 - s_1\right| = \left|\SI{19}{\m} - (\SI{-20}{\m})\right| = \SI{39}{\m}
	\end{equation}
	
	\paragraph{b)} Bei welcher(n) Frequenz(en) kommt es zu maximaler Verstärkung der Wellen im Nullpunkt? 
	\begin{equation}
		\Delta s = \SI{39}{\m} = k \cdot \lambda \quad (\text{mit } k \in \mathbb{N})
	\end{equation}
	
	Maximale gegenseitige Verstärkung bei Wellenlängen von $\lambda = \frac{\SI{39}{\m}}{k}$, wobei $k$ eine beliebige ganze Zahl ist.
	
	\newpage
	
	\subsection{Ungedämpfter Schwingkreis}
	
	\subsubsection{Herstellung Schwingkreis}
	Es soll ein Schwingkreis mit einer Eigenfrequenz von \SI{7,5}{\kilo\Hz} hergestellt werden.
	
	\paragraph{a)} Welche Kapazität muss ein Kondensator haben, wenn man eine Spule mit der Induktivität $L = \SI{0,25}{\H}$ verwendet?
	
	Formel für C aus Formel für Schwingungsdauer herleiten:
	\begin{equation}\label{formelc}
		\begin{split}
			T &= 2\pi \cdot \sqrt{LC} \\
			\Leftrightarrow LC &= \left(\frac{T}{2\pi}\right)^2 \\
			\Leftrightarrow C &= \frac{\left(\frac{T}{2\pi}\right)^2}{L} \\
			\Leftrightarrow C &= \frac{T^2}{4\pi^2 \cdot L}
		\end{split}
	\end{equation}
	
	T berechnen:
	\begin{equation}
		T = \frac{1}{f} = \frac{1}{\SI{7500}{\Hz}} = \SI{1,33e-4}{s}
	\end{equation}
	
	Werte in Formel aus \ref{formelc} eingesetzt:
	\begin{equation}
		C = \frac{\left(\SI{1,33e-4}{s}\right)^2}{4\pi^2 \cdot \SI{0,25}{\H}} = \SI{1,79e-9}{\farad} = \SI{1,79}{\nano\farad}
	\end{equation}
	
	\paragraph{b)} Welche Induktivität muss eine Spule haben, wenn man einen Kondensator mit der Kapazität $C = \SI{10}{\micro\farad}$ verwendet?
	
	Gleicher Ansatz wie in Herleitung \ref{formelc} liefert:
	\begin{equation}
		L = \frac{T^2}{4\pi^2 \cdot C}
	\end{equation}
	
	Mit eingesetzten Werten:
	\begin{equation}
		L = \frac{T^2}{4\pi^2 \cdot C} = \frac{\left(\SI{1,33e-4}{s}\right)^2}{4\pi^2 \cdot \SI{e-5}{\farad}} = \SI{0,337}{\H}
	\end{equation}
	
	\subsubsection{Frequenzen synchronisieren}
	An einer Feder, welche durch eine Kraft von \SI{4}{\newton} um \SI{2,4}{\centi\meter} gedehnt wird, hängt ein Körper mit der Masse \SI{0,5}{\kilogram}. In einem elektrischen Schwingkreis mit einer Induktivität von \SI{75}{\H} soll eine Schwingung erzeugt werden, die in ihrer Frequenz mit der Federschwingung übereinstimmt. Welche Kapazität ist erforderlich?
	
	Federhärte $D$ berechnen:
	\begin{equation}
		D = \frac{F}{s} = \frac{\SI{4}{\newton}}{\SI{0,024}{\m}} = \SI{166}{\newton\per\meter}
	\end{equation}
	
	Gleichsetzen der beiden Schwingungsdauern:
	\begin{equation}
		\begin{split}
			T_\text{Schwingk.} &= T_\text{Feder}\\
			\Leftrightarrow 2\pi \cdot \sqrt{LC} &= 2\pi \cdot \sqrt{\frac{m}{D}} \\
			\Leftrightarrow \sqrt{LC} &= \sqrt{\frac{m}{D}} \\
			\Leftrightarrow LC &= \frac{m}{D} \\
			\Leftrightarrow C &= \frac{m}{DL} \\
			\Leftrightarrow C &= \frac{\SI{0,5}{\kilogram}}{\SI{166}{\newton\per\meter} \cdot \SI{75}{\H}} = \SI{4,02e-5}{\farad} = \SI{40,2}{\micro\farad}
		\end{split}
	\end{equation}
	
	\newpage
	
	\subsection{Beugung am Gitter}
	
	\subsubsection{Gitter}
	Ein Gitter besitzt 100 Striche pro Millimeter.
	
	\paragraph{a)} Auf einem \SI{3}{\meter} entferntem Schirm beträgt der Abstand der beiden Maxima 2. Ordnung \SI{80}{\centi\meter}. Berechne die Wellenlänge des verwendeten Lichts.
	
	\begin{equation}
		\begin{split}
			\frac{k \cdot \lambda}{g} &= \sin\left(\arctan\left(\frac{a}{e}\right)\right) \\
			\Leftrightarrow \lambda &= \frac{\sin\left(\arctan\left(\frac{a}{e}\right)\right) \cdot g}{k} \\
			\Leftrightarrow \lambda &= \frac{\sin\left(\arctan\left(\frac{\SI{0,40}{\meter}}{\SI{3,00}{\meter}}\right)\right) \cdot \SI{e-5}{\meter}}{2} = \SI{6,61e-7}{\meter} = \SI{661}{\nm}
		\end{split}
	\end{equation}
	
	\paragraph{b)} Wie viele Stellen maximaler Helligkeit könnte man bei dieser Beugung insgesamt höchstens beobachten?
	
	\begin{equation}
		\begin{split}
			\sin\left(\alpha_k\right)=\frac{k \cdot \lambda}{g} &< 1 \\
			\Leftrightarrow k &< \frac{g}{\lambda} \\
			\Leftrightarrow k &< \frac{\SI{e-5}{\meter}}{\SI{661e-9}{\meter}} = 15,1
		\end{split}
	\end{equation}
	
	$k \in \mathbb{N} \Rightarrow k_{\text{max}} = 15$
	
	Maxima bis zur 15. Ordnung, also 31 Stellen maximaler Helligkeit.
	
	\subsubsection{Beugungsordnung}
	Stelle fest, welche Beugungsordnung mit unserem Gitter ($g^{-1} = \SI[per-mode=reciprocal]{300}{\per\meter}$) bei rotem Laserlicht ($\lambda = \SI{635}{\nm}$) maximal beobachtbar ist.
	
	\begin{equation}
	\begin{split}
		\sin\left(\alpha_k\right)=\frac{k \cdot \lambda}{g} &< 1 \\
		\Leftrightarrow k &< \frac{g}{\lambda} \\
		\Leftrightarrow k &< \frac{\SI{3,33e-6}{\meter}}{\SI{635e-9}{\m}} = 5,24
	\end{split}
	\end{equation}
	
	$k \in \mathbb{N} \Rightarrow k_{\text{max}} = 5$
	
	\section{Quanten und Atome}
	
	\subsection{Photoeffekt}
	
	\subsubsection{Halbierung}
	Beim Photoeffekt wird die Frequenz des einfallenden Lichtes halbiert. Verringert sich die kinetische Energie der Photoelektronen ebenfalls, auf den halben Wert, oder auf mehr oder weniger als den halben Wert? (mit Begründung)
	
	\begin{equation}
		\begin{split}
			W_\text{A} + E_\text{kin} &= h \cdot f \\
			E_\text{kin} &= h \cdot f - W_\text{A}
		\end{split}
	\end{equation}
	
	$\Rightarrow \frac{1}{2}E_\text{kin} = \frac{hf}{2} - \frac{W_\text{A}}{2}$
	
	\begin{equation}
		\begin{split}
			E'_\text{kin} &= h \cdot \frac{f}{2} - W_\text{A} \\
						  &= \frac{hf}{2} - W_\text{A}
		\end{split}
	\end{equation}
	
	
	
	Bei konstantem $W_\text{A}$ ist $E'_\text{kin} < \frac{1}{2}E_\text{kin}$.
	
	\subsubsection{Austrittsarbeit}
	Berechne für die Austrittsarbeit $W_\text{A} = \SI{0,8}{\eV}$ die Grenzfrequenz und die Grenzwellenlänge für das Auftreten von Photoelektronen! Welchem Spektralbereich gehört diese Frequenz an?
	
	$\SI{0,8}{\eV} = \SI{1,28e-19}{\J}$
	
	\begin{equation}
		\begin{split}
			W_\text{kin} &= hf - W_\text{A} \\
			0 &= \SI{6,63e-34}{\J\s} \cdot f - \SI{1,28e-19}{\J} \\
			f &= \frac{\SI{1,28e-19}{\J}}{\SI{6,63e-34}{\J\s}} = \SI{1,93e14}{\per\s}
		\end{split}
	\end{equation}
	\begin{equation}
		\lambda = \frac{c}{f} = \frac{\SI{3,00e8}{\m\per\s}}{\SI{1,93e14}{\Hz}} = \SI{1550}{\nm} \Rightarrow \text{Infrarotbereich}
	\end{equation}
	
	\subsubsection{Kinetische Energie}
	Welche kinetische Energie (in eV) erhält ein Photoelektron, das von Licht der Wellenlänge $\SI{400}{\nm}$ aus einem Metall mit $\SI{0.8}{\eV}$ ausgelöst wird? Berechne seine Geschwindigkeit!
	
	\begin{equation}
		\begin{split}
			E_\text{kin} &= hf - W_\text{A} \\
			&= \SI{4,14e-15}{\eV\s} \cdot \frac{\SI{3,00e8}{\m\per\s}}{\SI{400e-9}{\m}} - \SI{0,8}{\eV} \\
			&= \SI{2,31}{\eV} = \SI{3,70e-19}{\J}
		\end{split}
	\end{equation}
	
	\begin{equation}
		\begin{split}
			E_\text{kin} &= \frac{1}{2} m_\text{e}v^2 \\
			v &= \sqrt{\frac{2E_\text{kin}}{m_\text{e}}} \\
			  &= \sqrt{\frac{2 \cdot \SI{3,70e-19}{\J}}{\SI{9,11e-31}{\kg}}} \\
			  &= \SI{9,01e5}{\m\per\s}
		\end{split}
	\end{equation}
	
	\newpage
	
	\subsection{Masse und Impuls des Photons}
	
	\subsubsection{Hg-Spektrum}
	Das gelbe Licht im Hg-Spektrum hat eine Wellenlänge von \SI{579}{\nm}. Berechne die Energie, die Masse und den Impuls eines Photons aus diesem Licht!
	
	$f = \frac{c}{\lambda} = \frac{\SI{3,00e8}{\m\per\s}}{\SI{579e-9}{\m}} = \SI{5,18e14}{\per\s}$
	
	\begin{equation}
		\begin{split}
			E_\text{Ph} &= h \cdot f \\
						&= \SI{6,63e-34}{\J\s} \cdot \SI{5,18e14}{\per\s} \\
						&= \SI{3,43e-19}{\J}
		\end{split}
	\end{equation}
	
	\begin{equation}
		\begin{split}
			m_\text{Ph} &= \frac{h \cdot f}{c^2} \\
						&= \frac{\SI{6,63e-34}{\J\s} \cdot \SI{5,18e14}{\per\s}}{\left(\SI{3,00e8}{\m\per\s}\right)^2} \\
						&= \SI{3,82e-36}{\kg}
		\end{split}
	\end{equation}
	
	\begin{equation}
		\begin{split}
			p_\text{Ph} &= \frac{h}{\lambda} \\
						&= \frac{\SI{6,63e-34}{\J\s}}{\SI{579e-9}{\m}} \\
						&= \SI{1,15e-27}{\J\s\per\m}
		\end{split}
	\end{equation}
	
	\newpage
	
	\subsection{Materiewellen}
	
	\subsubsection{\textsc{D\lowercase{e}-B\lowercase{roglie}}-Wellenlänge}
	Wie groß ist die \textsc{De-Broglie}-Wellenlänge eines Elektrons, das eine Beschleunigungsspannung von \SI{1}{\volt}, von \SI{10}{\volt} oder von \SI{100}{\volt} durchlaufen hat?
	
	\begin{equation}
		\lambda_\text{e1} = \frac{h}{\sqrt{\SI{2}{eV} \cdot m_\text{e}}} = \SI{1,23}{\nm}
	\end{equation}
	
	\begin{equation}
		\lambda_\text{e10} = \frac{h}{\sqrt{\SI{20}{eV} \cdot m_\text{e}}} = \SI{0,388}{\nm}
	\end{equation}
	
	\begin{equation}
		\lambda_\text{e100} = \frac{h}{\sqrt{\SI{200}{eV} \cdot m_\text{e}}} = \SI{0,123}{\nm}
	\end{equation}
	
	\newpage

	\subsection{Aufgaben zu Serienformeln}
	Die Funktionsweise des Rubinlasers basiert auf den in \ref{fig:rublas} dargestellten Energiezuständen der im Rubinkristall enthaltenen Cr$^{\text{3+}}$-Ionen. Durch Wechselwirkung mit den von der Blitzlampe gelieferten Photonen werden die Ionen aus ihrem Energiezustand $\text{W}_1$ auf die Energieniveaus $\text{W}_3$ bzw. $\text{W}_4$ angehoben, von wo aus sie auf das Niveau $\text{W}_2$ wechseln. Bei der Rückkehr vom Niveau $\text{W}_2$ in den Grundzustand $\text{W}_1$ wird rotes Laserlicht emittiert.
	
	\begin{figure}[ht]
		\centering
		\begin{tikzpicture}[decoration = {snake,   % <-- added
				pre length=2pt,post length=4pt}]
			% Rahmen 
			\draw[thick,->] (0,0) -- (0,4);
			
			\draw[thick] (-.1,0) -- (.1,0);
			\draw[thick,dotted] (.1,0) -- (.5,0);
			\draw[thick] (.5,0) -- (6,0);
			
			% Inhalt
			\draw[thick] (-.1,1.79) -- (.1,1.79);
			\draw[thick,dotted] (.1,1.79) -- (3,1.79);
			\draw[thick] (3,1.79) -- (6,1.79);
			
			\draw[thick] (-.1,2.22) -- (.1,2.22);
			\draw[thick,dotted] (.1,2.22) -- (.5,2.22);
			\draw[thick] (.5,2.22) -- (2.75,2.22);
			
			\draw[thick] (-.1,3.03) -- (.1,3.03);
			\draw[thick,dotted] (.1,3.03) -- (.5,3.03);
			\draw[thick] (.5,3.03) -- (2.75,3.03);
			
			% Gerade Pfeile
			\draw[thick,->] (1.5,0) -- (1.5,3.03);
			\draw[thick,->] (2.25,0) -- (2.25,2.22);
			\draw[thick,->] (5,1.79) -- (5,0);
			
			% Wellenpfeile
			\draw[thick,decorate,->] (.25,.645) -- (2.25,.645);
			\draw[thick,decorate,->] (.25,.945) -- (1.5,.945);
			
			\draw[thick,decorate,->] (1.65,2.93) -- (5,1.79);
			\draw[thick,decorate,->] (2.4,2.12) -- (3.36973684211,1.79);
			
			\draw[thick,decorate,->] (5,.795) -- (7,.795);
			
			% Text
			\draw (-.247,0) node[font=\tiny] {0};
			\draw (-.4,1.79) node[font=\tiny] {1.79};
			\draw (-.4,2.22) node[font=\tiny] {2.22};
			\draw (-.4,3.03) node[font=\tiny] {3.03};
			
			\draw (0,4.2) node[font=\tiny] {$W$ in \SI{}{\eV}};
			\draw (5,2.625) node[font=\tiny] {Energieabgabe an Kristall};
			\draw (.8,1.2) node[font=\tiny] {Absorption};
			\draw (6,1) node[font=\tiny] {Emission};
			
			\draw (.65,.15) node[font=\tiny] {$W_1$};
			\draw (5.8,1.94) node[font=\tiny] {$W_1$};
			\draw (.65,2.37) node[font=\tiny] {$W_3$};
			\draw (.65,3.18) node[font=\tiny] {$W_4$};
						
		\end{tikzpicture}
		\caption{Rubinlaser}
		\label{fig:rublas}
	\end{figure}
		
	\paragraph{Bestätigen} Sie anhand von \ref{fig:rublas}, dass die Energiedifferenz zwischen $W_1$ und $W_2$ den Betrag \SI{2,86e-19}{\J} hat.
		
	\begin{equation}
		\Delta W = W_2 -W_1 = \SI{1.79}{\eV} - \SI{0}{\eV} = \SI{1.79}{\eV}
	\end{equation}
		
	\begin{equation}
		\SI{1.79}{\eV} \cdot \SI{1.60e-19}{\J\per\eV} = \SI{2,86e-19}{\J} \qed
	\end{equation}

	\paragraph{Berechnen} Sie die Wellenlänge des emittierten Laserlichts.
	
	\begin{equation}
		\begin{split}
			h \cdot f &= W_2 -W_1 \\
			\Leftrightarrow f &= \frac{W_2 -W_1}{h} \\
			&= \frac{\SI{2,86e-19}{\J}}{h} \\
			&= \SI{4,32e14}{\s}
		\end{split}
	\end{equation}
	
	\begin{equation}
		\lambda = \frac{c}{f} = \frac{c}{\SI{4,32e14}{\s}} = \SI{695}{\nm}
	\end{equation}
	
	

\end{document}
